\documentclass[]{article}
\usepackage{lmodern}
\usepackage{amssymb,amsmath}
\usepackage{ifxetex,ifluatex}
\usepackage{fixltx2e} % provides \textsubscript
\ifnum 0\ifxetex 1\fi\ifluatex 1\fi=0 % if pdftex
  \usepackage[T1]{fontenc}
  \usepackage[utf8]{inputenc}
\else % if luatex or xelatex
  \ifxetex
    \usepackage{mathspec}
  \else
    \usepackage{fontspec}
  \fi
  \defaultfontfeatures{Ligatures=TeX,Scale=MatchLowercase}
\fi
% use upquote if available, for straight quotes in verbatim environments
\IfFileExists{upquote.sty}{\usepackage{upquote}}{}
% use microtype if available
\IfFileExists{microtype.sty}{%
\usepackage{microtype}
\UseMicrotypeSet[protrusion]{basicmath} % disable protrusion for tt fonts
}{}
\usepackage[margin=1in]{geometry}
\usepackage{hyperref}
\hypersetup{unicode=true,
            pdftitle={Problem Set 1},
            pdfauthor={ECON 326 - Industrial Organization - Spring 2020},
            pdfborder={0 0 0},
            breaklinks=true}
\urlstyle{same}  % don't use monospace font for urls
\usepackage{graphicx}
% grffile has become a legacy package: https://ctan.org/pkg/grffile
\IfFileExists{grffile.sty}{%
\usepackage{grffile}
}{}
\makeatletter
\def\maxwidth{\ifdim\Gin@nat@width>\linewidth\linewidth\else\Gin@nat@width\fi}
\def\maxheight{\ifdim\Gin@nat@height>\textheight\textheight\else\Gin@nat@height\fi}
\makeatother
% Scale images if necessary, so that they will not overflow the page
% margins by default, and it is still possible to overwrite the defaults
% using explicit options in \includegraphics[width, height, ...]{}
\setkeys{Gin}{width=\maxwidth,height=\maxheight,keepaspectratio}
\IfFileExists{parskip.sty}{%
\usepackage{parskip}
}{% else
\setlength{\parindent}{0pt}
\setlength{\parskip}{6pt plus 2pt minus 1pt}
}
\setlength{\emergencystretch}{3em}  % prevent overfull lines
\providecommand{\tightlist}{%
  \setlength{\itemsep}{0pt}\setlength{\parskip}{0pt}}
\setcounter{secnumdepth}{0}
% Redefines (sub)paragraphs to behave more like sections
\ifx\paragraph\undefined\else
\let\oldparagraph\paragraph
\renewcommand{\paragraph}[1]{\oldparagraph{#1}\mbox{}}
\fi
\ifx\subparagraph\undefined\else
\let\oldsubparagraph\subparagraph
\renewcommand{\subparagraph}[1]{\oldsubparagraph{#1}\mbox{}}
\fi

%%% Use protect on footnotes to avoid problems with footnotes in titles
\let\rmarkdownfootnote\footnote%
\def\footnote{\protect\rmarkdownfootnote}

%%% Change title format to be more compact
\usepackage{titling}

% Create subtitle command for use in maketitle
\providecommand{\subtitle}[1]{
  \posttitle{
    \begin{center}\large#1\end{center}
    }
}

\setlength{\droptitle}{-2em}

  \title{Problem Set 1}
    \pretitle{\vspace{\droptitle}\centering\huge}
  \posttitle{\par}
    \author{ECON 326 - Industrial Organization - Spring 2020}
    \preauthor{\centering\large\emph}
  \postauthor{\par}
      \predate{\centering\large\emph}
  \postdate{\par}
    \date{Due by Thursday, February 27, 2020}

\usepackage{amsmath, tikz}
\usepackage{multirow, multicol, booktabs}

\begin{document}
\maketitle

Please write your answers to the following questions on a piece of
paper, or download and print a PDF copy to write on. You may also type
your answers and print out a hard copy.

Answer all of the following questions briefly (1-3 sentences). Use
examples as necessary. Be sure to label graphs fully, if appropriate.
For calculation problems, please show all work. Simply writing the
answer, even if correct, may result in loss of points.

You may work together (and I highly encourage that) but you must turn in
your own answers. Your TA, under my supervision, will grade homeworks
70\% for completion, and for the remaining 30\%, pick one question to
grade for accuracy - so it is best that you try every problem, even if
you are unsure how to complete it accurately.

\hypertarget{competitive-markets}{%
\section{Competitive Markets}\label{competitive-markets}}

\begin{enumerate}
\def\labelenumi{\arabic{enumi}.}
\tightlist
\item
  In a competitive industry, why are economic profits normal (zero) in
  the long run? What about if firms are not identical, and have
  different costs?
\end{enumerate}

\clearpage

\begin{enumerate}
\def\labelenumi{\arabic{enumi}.}
\setcounter{enumi}{1}
\tightlist
\item
  Assume that consumers view tax preparation services as
  undifferentiated among producers, and that there are hundreds of
  companies offering tax preparation. The current market equilibrium
  price is \$120. Amy's Audits is a local tax preparation firm that has
  a daily short-run cost structure given by:
\end{enumerate}

\[\begin{aligned}
C(q)&=100+4q^2\\
MC(q)&=8q\\
\end{aligned}\]

where \(q\) is the number of tax returns per day.

\begin{enumerate}
\def\labelenumi{\alph{enumi}.}
\tightlist
\item
  How many tax returns should Amy prepare each day if her goal is to
  maximize profits?
\item
  How much profit will she earn each day?
\item
  At what market price would she break even?
\item
  Below what hypothetical price would she shut down in the short run?
\item
  Sketch a graph and be sure to label everything you found in parts A-D.
\item
  What is Amy's supply curve in the \emph{short run}? Write a function
  or describe it via the graph in E.
\item
  What is Amy's supply curve in the \emph{long run}? Write a function or
  describe it via the graph in E.
\end{enumerate}

\clearpage

\hypertarget{monopoly}{%
\section{Monopoly}\label{monopoly}}

\begin{enumerate}
\def\labelenumi{\arabic{enumi}.}
\setcounter{enumi}{2}
\tightlist
\item
  What is the difference between allocative efficiency, productive
  efficiency, and X-efficiency?
\end{enumerate}

\vspace{3in}

\begin{enumerate}
\def\labelenumi{\arabic{enumi}.}
\setcounter{enumi}{3}
\tightlist
\item
  Sketch a graph of a monopoly with no fixed costs, and constant
  equivalent average \& marginal costs. Be sure to label all of the
  following:
\end{enumerate}

\begin{itemize}
\tightlist
\item
  The equilibrium quantity and price if the market were competitive
\item
  The profit-maximizing quantity and price for the monopoly
\item
  The consumer surplus, producer surplus, and deadweight loss under
  monopoly
\end{itemize}

\clearpage

\begin{enumerate}
\def\labelenumi{\arabic{enumi}.}
\setcounter{enumi}{4}
\tightlist
\item
  Explain what the goal of price discrimination is for a firm.\footnote{Yes,
    we know firms aim to maximize profits, but \emph{how} does price
    discrimination assist in acheiving this goal?} What are the
  conditions required for a firm to engage in price discrimination? What
  are the different types of price discrimination, and how does each
  work?
\end{enumerate}

\clearpage

\begin{enumerate}
\def\labelenumi{\arabic{enumi}.}
\setcounter{enumi}{5}
\tightlist
\item
  You are a monopoly producer of tablets. You have a cost structure:
\end{enumerate}

\[\begin{aligned}
C(q)&=10q^2+200q+1000\\
MC(q)&=20q+200\\
\end{aligned}\]

You estimate the demand for your tablets to be: \[q=100-0.2p\] where
\(q\) is millions of tablets.

\begin{enumerate}
\def\labelenumi{\alph{enumi}.}
\tightlist
\item
  Find the function for your marginal revenues.
\item
  How many tablets should you produce, and at what price, to maximize
  your profit?
\item
  What is the cost per tablets at the quantity you are producing?
\item
  What is your total profit?
\item
  What would be the lowest possible price you would need to charge to
  break even?
\item
  How much of your price is markup over marginal cost?
\item
  Calculate the elasticity of demand at your profit-maximizing price.
\end{enumerate}

\clearpage

\begin{enumerate}
\def\labelenumi{\arabic{enumi}.}
\setcounter{enumi}{6}
\tightlist
\item
  Consider a boat rental firm on a popular beach that has a constant
  average and marginal cost of \$30 per boat hire. It has estimated that
  demand from Locals \((L)\) and demand from Tourists \((T)\) are:
  \[\begin{aligned}
  q_L&=40-0.4p\\
  q_T&=25-0.1p\\
  \end{aligned}\]
\end{enumerate}

\begin{enumerate}
\def\labelenumi{\alph{enumi}.}
\tightlist
\item
  Suppose it must charge a single price to all customers. Find the
  profit-maximizing quantity, price, and the total profits.
\item
  How much of the price is markup?
\item
  What is the price elasticity of demand at this price?
\item
  Now suppose the firm is able to segment the market and charge
  different prices to Tourists and Locals. Find the profit-maximizing
  quantity, price, and the total profits.
\item
  For each segment of the market: how much of the price is markup, and
  what is the price elasticity of demand at the optimal price? How did
  the price for each segment change from the single price (Part A), and
  why?
\end{enumerate}

\clearpage

\hypertarget{factor-markets-monopsony}{%
\section{Factor Markets \& Monopsony}\label{factor-markets-monopsony}}

\begin{enumerate}
\def\labelenumi{\arabic{enumi}.}
\setcounter{enumi}{7}
\tightlist
\item
  Carl's Coal Mining operates in a remote area. Because of its location,
  it has monopsony power in the local labor market for miners. Its
  marginal revenue product of labor is \[MRP_L = 400-5L\] where \(L\) is
  the total number of miners. The labor supply curve of local miners is
  \[w = 5L-50\] where \(w\) is the wage (in \$1000's per miner).
\end{enumerate}

\begin{enumerate}
\def\labelenumi{\alph{enumi}.}
\tightlist
\item
  Write a function for the marginal cost of labor.
\item
  What quantity of workers will the mine hire, and what wage will it pay
  its workers?
\item
  What would the quantity of workers be, and what would the wage be, if
  there was competition among other local mines for labor?
\item
  Sketch a graph of this market, and be sure to label all of your
  findings (and show the Deadweight Loss) from Parts A-C.
\end{enumerate}


\end{document}
